% chktex-file 44

\section{Parameter Descriptions}

The parameters for \textbf{Gpio} are shown below in
Table 3.

\renewcommand*{\arraystretch}{1.4}
\begingroup
\small
\rowcolors{2}{gray!30}{gray!10} % Alternating colors start from the second row
\arrayrulecolor{gray!50}
\begin{longtable}[H]{
    | p{0.25\textwidth}
    | p{0.10\textwidth}
    | p{0.05\textwidth}
    | p{0.05\textwidth}
    | p{0.47\textwidth} |
  }
  \hline
  \rowcolor{dark-gray}

  \textcolor{white}{\textbf{Name}} &
  \textcolor{white}{\textbf{Type}} &
  \textcolor{white}{\textbf{Min}} &
  \textcolor{white}{\textbf{Max}} &
  \textcolor{white}{\textbf{Description}} \\ \hline
  \endfirsthead

  \textcolor{white}{\textbf{Name}} &
  \textcolor{white}{\textbf{Type}} &
  \textcolor{white}{\textbf{Min}} &
  \textcolor{white}{\textbf{Max}} &
  \textcolor{white}{\textbf{Description}}            \\ \hline
  \endhead


  \endfoot

  dataWidth   &
  Int       &
  1         &
  $\leq$ 32          &
  The data width of Gpio ports, PWDATA, and PRDATA. Can be 8, 6, or 32 bits wide \\ \hline

  addrWidth     &
  Int           &
  1             &
  $\leq$ 32       &
  The Apb address bus width  \\ \hline

\end{longtable}
\captionsetup{aboveskip=0pt}
\captionof{table}{Parameter Descriptions}\label{table:params}
\endgroup

The Gpio is instantiated into a design as follows:

\begin{lstlisting}[language=Scala]

  // Valid Gpio Instantiation Example
  val myGpio = new Gpio(
    dataWidth = 32, 
    addrWidth = 32 ) 

  \end{lstlisting}
