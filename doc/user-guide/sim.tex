\section{Simulation}

\subsection{Tests}
The test bench generates a number (default is 2) configurations of the
GPIO that are highly randomized. The user can run the full test-suite  
with n-configurations using "sbt "test" -DtestName="regression"".

User can also run individual tests or test groups by substituting "regression" with
the relevant test name. A description of the tests is below:

\begin{table}[h]
  \resizebox{\textwidth}{!}{
  \centering
  \begin{tabular}{|c|c|c|c|}
      \hline
      \rowcolor{dark-gray}  % Dark grey background for header row
      \textcolor{white}{\textbf{Test Group Name}} & \textcolor{white}{\textbf{Test Name}} & \textcolor{white}{\textbf{Test Type}} & \textcolor{white}{\textbf{Test Description}} \\ \hline
      basicRegisterRW & directionRegister & Directed & Writes and Reads Direction Register w/ Random Data \\ \hline
      basicRegisterRW & modeRegister & Directed & Writes and Reads Mode Register w/ Random Data \\ \hline
      basicRegisterRW & outputRegister & Directed & Writes and Reads Output Register w/ Random Data \\ \hline
      basicRegisterRW & inputRegister & Directed & Writes and Reads Input Register w/ Random Data \\ \hline
      basicRegisterRW & invalidAddress  & Directed & Writes and Reads from Invalid Address Space \\ \hline
      maskingRegisters & maskingAnd & Directed & Atomic Register "AND" Operation \\ \hline
      maskingRegisters & maskingAndMode & Random & Set "AND" Atomic Register, Configure Mode, Test Output\\ \hline
      interruptTriggers & triggerHigh & Directed & Test Interrupt Trigger on Level High \\ \hline
      interruptTriggers & triggerLow & Directed & Test Interrupt Trigger on Level Low\\ \hline
      interruptTriggers & triggerRising & Directed & Test Interrupt Trigger on Rising Edge\\ \hline
      interruptTriggers & triggerFalling & Directed & Test Interrupt Trigger on Falling Edge \\ \hline
      interruptTriggers & combinedTriggerLevel & Random & Test All Interrupt Triggers at Same Time on Different Ports \\ \hline
      virtualPorts & virtualMapping & Directed & Test Virtual Port to Physical Port Mapping\\ \hline
      virtualPorts & virtualWriting & Directed & Writing to a Virtual Port w/ Random Data\\ \hline
      virtualPorts & virtualToPhysical & Directed & Verify Physcial Port Output after Virtual Port Write\\ \hline
      virtualPorts & disableVirtual & Directed & Disable Virtual Port and Verify Physical Port Output \\ \hline
      virtualPorts & invalidVirtual & Directed & Invalid Virtual Port Mapping \\ \hline
      virtualPorts & disabledVirtualRead & Directed & Read from Disabled Virtual Port \\ \hline
      virtualPorts & overlappingVirtualPorts & Random & Overlaping Virtual Ports Mapped to the Same Physical Port \\ \hline
      virtualPorts & virtualInput & Directed & Virtual Port as Input \\ \hline
      modeOperation & pushPullMode & Random & Write to random ports, set random ports to PPL, test output \\ \hline
      modeOperation & drainMode & Random & Write to random ports, set random ports to open drain mode, test output \\ \hline
  \end{tabular}
  }
  \caption{Test Suite}
\end{table}


\subsection{Toggle Coverage}
Current Score: 100\%

All inputs and outputs are checked to insure each toggle at least once. If coverage is enabled
during core instantiation, a cumulative coverage report of all tests is generated under ./out/cov/verliog.

Exceptions are higher bits of the \emph{PADDR}, \emph{PWDATA}, and \emph{PRDATA}
which are intended to be static during each simulation. These signals are
excluded from coverage checks.

\subsection{Code Coverage}
Current Score: 82.17\%

Code coverage for all tests can be generated as follows:
\begin{verbatim}
  $ sbt coverage test -DtestName="allTests"
  $ sbt coverageReport
  $ python3 -m http.server 8000 --directory target/scoverage-report/
  View report on local host: http://localhost:8000/index.html
\end{verbatim}

\subsection{Running simulation}

Simulations can be run directly from the command prompt as follows:

\begin{verbatim}
  $ sbt test -DtestName="allTests"
\end{verbatim}

or from make as follows:

\texttt{\$ make test}

\subsection{Viewing the waveforms}

The simulation generates an FST file that can be viewed using a waveform viewer. The command to view the waveform is as follows:
\begin{verbatim}
  $ gtkwave ./out/test/Gpio.fst
\end{verbatim}
