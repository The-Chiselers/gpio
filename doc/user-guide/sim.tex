\section{Simulation}

\subsection{Tests}
The test bench generates a number (default is 50) configurations of the
DynamicFifo that are highly randomized. There are two flavors of tests:

\begin{itemize}
  \item {Directed tests that fill the FIFO with random data and then read back
        the results to verify that the read data matches the writted data.}
  \item {Lengthy random tests that are used to check odd combinations of
        configurations and to compile code coverage data.}
\end{itemize}

\subsection{Code coverage}
All inputs and outputs are checked to insure each toggle at least once. An error
will be thrown in case any port fails to toggle.

The only exception are the \emph{almostEmptyLevel} and \emph{almostFullLevel}
which are intended to be static during each simulation. These signals are
excluded from coverage checks.

\subsection{Running simulation}

Simulations can be run directly from the command prompt as follows:

\begin{verbatim}
  $ sbt "test"
\end{verbatim}

or from make as follows:

\texttt{\$ make test}

\subsection{Viewing the waveforms}

The simulation generates an FST file that can be viewed using a waveform viewer. The command to view the waveform is as follows:
\begin{verbatim}
  $ gtkwave ./out/test/GPIO.fst
\end{verbatim}
