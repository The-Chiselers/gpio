\section{Theory of Operations}

\subsection{Introduction}
The \textbf{GPIO} module is configurable with parameters for setting data and address widths. It interfaces with the CPU via the APB3 bus. The signals \textbf{gpioInput}, \textbf{gpioOutput} and \textbf{gpioOutputEnable} are designed to interface with external hardware, such as a tri-state buffer, to facilitate bi-directional communication on physical pins.

\begin{figure}[h]
  \includegraphics[width=0.90\textwidth]{images/block-diagram-gpio.png}
  \caption{GPIO Block Diagram}\label{fig:block-diagram}
\end{figure}

% It features the following status flags which are described in
% Table~\ref{table:ports}.

% \begin{itemize}[noitemsep]
%   \item{empty}
%   \item{full}
%   \item{almostEmpty}
%   \item{almostFull}
% \end{itemize}

% When \textit{push} is asserted, the data on the \textit{dataIn} port is enqued
% on the next rising edge of \textit{clock}. When \textit{pop} is asserted, the
% top of the FIFO is dequed and immediately available on the \textit{dataOut}
% port. Pop and Push operations can be simulataneous.

% There are two error conditions which produce the following effects:
% \begin{itemize}
%   \item{When \textit{pop} is asserted and the FIFO is empty (\textit{empty} is
%         active), \textit{dataOut} will contain the last valid data held in
%         the FIFO.}
%   \item{When \textit{push} is asserted and the FIFO is full (\textit{full} is
%         active), \textit{dataIn} will be ignored and not enqued.}
% \end{itemize}

% The \textit{almostEmpty} and \textit{almostFull} flags allow for additional
% feedback to the system that is useful for optimizing data flow control. The
% levels of these flags can be programmed dynamically through the
% \textit{almostEmptyLevel} and \textit{almostFullLevel} ports.

\newpage
\subsection{Interface Timing}

GPIO has a synchronous APB3 interface, and a GPIO interface. The timing diagram shown below
in Figure~\ref{fig:timing} represents an instantiation with the following
parameters.

\begin{lstlisting}[language=Scala]
val myGPIO = new GPIO(
  dataWidth = 16, 
  addrWidth = 16 ) 
\end{lstlisting}

\begin{figure}[h]
  \includegraphics[width=\textwidth]{images/GPIO-timing.png}
  \caption{Timing Diagram}\label{fig:timing}
\end{figure}

This shows the operation of the basic read/write register operations following the APB protocol. Registers DIRECTION, MODE, and OUTPUT are
written to, while register INPUT is read from.

The \textit{gpioOutputEnable} port is driven to a value of 0x000F after 0x000F is written to the DIRECTION register 
at address 0x00. Next, the MODE register at address 0x06 is written a value of 0x0009. The \textit{gpioOutput} port 
then has a value of 0x0009 because of the open-drain mode operation. 

The OUTPUT register is then written a value of 0x0006 at address 0x02. \textit{gpioOutput} takes on a value of 0x000F
since the MSB and LSB are operating on open-drain mode, and the middle two bits are operating on push-pull mode. 

The INPUT register is read from at address 0x04, which has a value of 0x0070 since \textit{gpioInput} has a value fo 0x0070.
On the final transaction, PSLVERR goes high because an invalid address is written to.
