\section{Usage Examples}

\subsection{Direction Register Configuration}
The following example demonstrates how to configure GPIO pin directions:

\begin{verbatim}
// 1. Configure GPIO pins 0-7 as outputs, 8-15 as inputs
write(GPIO_DIRECTION_0, 0x00FF);    // Lower byte = outputs
write(GPIO_DIRECTION_1, 0xFF00);    // Upper byte = inputs

// 2. Verify configuration
uint16_t dir0 = read(GPIO_DIRECTION_0);
uint16_t dir1 = read(GPIO_DIRECTION_1);

if ((dir0 != 0x00FF) || (dir1 != 0xFF00)) {
    // Configuration error
    // ...
    return;
}
\end{verbatim}

\subsection{Output Register Operation}
The following example demonstrates writing to output pins:

\begin{verbatim}
// 1. Set pins 0-3 high, others low (assuming pins are configured as outputs)
write(GPIO_OUTPUT_0, 0x000F);

// 2. Toggle output pins
uint16_t current = read(GPIO_OUTPUT_0);
write(GPIO_OUTPUT_0, ~current & 0xFFFF);

// 3. Write to specific pin (bit 5)
uint16_t new_value = (read(GPIO_OUTPUT_0) | (1 << 5));
write(GPIO_OUTPUT_0, new_value);
\end{verbatim}

\subsection{Input Register Operation}
The following example demonstrates reading input pins:

\begin{verbatim}
// 1. Read all input pins (assuming pins are configured as inputs)
uint16_t inputs = read(GPIO_INPUT_0);

// 2. Check specific pin state (bit 7)
if (inputs & (1 << 7)) {
    // Pin is high
    // ...
} else {
    // Pin is low
    // ...
}

// 3. Wait for pin state change
uint16_t previous = read(GPIO_INPUT_0);
while ((read(GPIO_INPUT_0) & 0x0100) == (previous & 0x0100)) {
    // Wait for bit 8 to change
}
\end{verbatim}

\subsection{Mode Register Configuration}
The following example demonstrates configuring pin modes:

\begin{verbatim}
// 1. Configure pins 0-3 as push-pull, 4-7 as open-drain
write(GPIO_MODE_0, 0x00F0);    // 00001111 (0-3: push-pull, 4-7: open-drain)

// 2. Configure pins 8-11 with pull-up, 12-15 with pull-down
write(GPIO_MODE_1, 0x0F00);    // 00001111 (8-11: pull-up, 12-15: pull-down)

// 3. Verify configuration
if ((read(GPIO_MODE_0) != 0x00F0) || (read(GPIO_MODE_1) != 0x0F00)) {
    // Handle configuration error
    // ...
}
\end{verbatim}

\subsection{Error Handling}
The following example demonstrates invalid address handling:

\begin{verbatim}
// 1. Attempt to write to invalid address
uint32_t status = write(INVALID_ADDR, 0x1234);

// 2. Check for error
if (status & ERROR_BIT) {
    // Handle invalid address error
    printf("Error: Invalid address accessed\n");
    // ...
}

// 3. Attempt to read from invalid address
uint16_t data = read(INVALID_ADDR);
if (data != 0) {
    // Handle unexpected data
    // ...
}
\end{verbatim}